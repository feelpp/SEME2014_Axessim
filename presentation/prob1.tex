\begin{frame}
\frametitle{Sous forme de matrice}
%Consid\`ere le mod\`ele
%\begin{equation}
%-\Delta\varphi = j_z
%\end{equation}
On peut \'ecrire
 \begin{equation}
   \tilde\varphi = \sum_k \phi_k \tilde{\varphi}_k
 \end{equation}
 avec $\tilde{\varphi}$ est solution de probl\`eme suivante
 \begin{equation}
   \label{eq:2}
 \begin{cases}
 -\Delta{\varphi} = 0 \quad \text{sur } \Omega \\
 {\varphi} = \delta_{ij}
 \end{cases}
 \end{equation}
 donc on a
 \begin{equation}
 \sum_j \phi_j \left( \int_{w_i}-\Delta\tilde{\varphi}_j \right) =I_i
 \end{equation}
 ou bien
 \begin{equation}
  \sum_j \phi_j \left( \int_{\partial w_i}-\nabla\tilde{\varphi}_j \right) \cdot n =I_i
 \end{equation}
 en d\'eduit
  \begin{equation}
  \sum_j M_{ij}\phi_j =I_i \quad \forall i
 \end{equation}
%
%\begin{eqnarray*}
%\frac{\partial U}{\partial z} =ZI, \quad Z=R+j\omega L,\\
%\frac{\partial I}{\partial z} =YU, \quad Y=G+j\omega C.
%\end{eqnarray*}


\end{frame}

 \begin{frame}
\frametitle{Calcul la matrice M}
On a
\begin{equation}
  \label{eq:3}
  M_{ij} = \int_{\partial w_j} \frac{\partial\varphi_i}{\partial n}
\end{equation}
 Properties de la matrice $M$
 \begin{itemize}
 \item Sym\'etrique
 \item D\'efinie positive
 \item Inversible
 \end{itemize}
 Ou bien
\begin{equation}
  \label{eq:4}
  M\phi =I \quad \text{ou bien } \phi =LI \quad \text{o\`u } L=M^{-1}
\end{equation}


\end{frame}

\begin{frame}
\frametitle{Calcul la matrice M}
Probl\`eme quand on a calculer directement l'int\'egrale : La matrice $M$ n'est plus sym\'etrique
=> solution: formulation faiblement
\begin{equation}
\int_w -\Delta\varphi \psi =\int_w \nabla\varphi\nabla\psi - \int_{\partial w} \nabla\varphi\cdot n \psi =\int_{\partial w} \nabla\varphi\cdot n
\end{equation}
avec $\psi \in H^1(w)$ satifait
\begin{equation}
\psi(z) =
\begin{cases}
0 \qquad \text{ si } z\in w \\
1 \qquad \text{ si } z\in \partial w
\end{cases}
\end{equation}


\end{frame}
