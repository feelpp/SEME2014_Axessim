%%%%%cas simple d'un blindage avec des conducteurs %%%%%%%%%
\begin{frame}
\frametitle{Cas simple d'un blindage avec des conducteurs}
\begin{columns}[T]
\column{0.45\linewidth}
\begin{center}
\includegraphics[scale=0.2]{figures/f3}
\end{center}
\column{0.55\linewidth}
Blindage de r\'er\'erence $w_0$, $2$ conducteurs $w_1, w_2$
\begin{itemize}
\item $1^{ere}$ \'etape : calcul $\varphi_i$ avec $i=1,2$-solution de
 \begin{equation}
 \begin{cases}
 -\Delta\varphi_i = 0 \quad \text{sur } w0 \\
 \varphi_i =  
 \begin{cases}
 1 \quad \text{sur } w_i \\
 0 \quad \text{sur} w
 \end{cases}
 \end{cases}
 \end{equation}
\item Chaque conducteur $w_i$ a $NN_i$ sous conducteurs dedans
\item Chaque conducteur $w_i$ matrice d'inductance $L_{int}^i$
\item $L_{ext}$ : matrice d'inductance des conducteurs ext\'erieurs  
\end{itemize}
\end{columns}
\end{frame} 

%%%%%cas simple des blindage avec des conducteurs (2 niveaux) %%%%%%%%%