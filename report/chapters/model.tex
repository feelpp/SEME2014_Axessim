\subsection*{The governing equation}
\todo[inline]{from helluy's doc}
The transmission model line is given by the following system of differential equations
\begin{equation}
  \label{eq:model}
    \begin{array}{ccc}
      \partial_z U&=& L\partial_t I + RI \\
      \partial_z I&=& C\partial_t U + GU \\
    \end{array}
\end{equation}

where we denoted the following square matrices:
\begin{itemize}
\item  $L$ the inductance, required to be \textbf{definite-positive and symmetric}
\item $C$ the capacity, defined as the inverse of $L$, required to be \textbf{definite positive and symmetric}
\item $R$ the resistivity\todo{pas sur du mot}, assumed to be diagonal
\item $G$ the conductancy, defined as the inverse of $R$.
\end{itemize}

we get rid \todo{change this} of the time dependancy in $\eqref{eq:model}$ by applying a Fourier transformation $(U,I) \mapsto (\hat{U},\hat{I})$, one finally gets, after dropping the hats the following system
\begin{equation}
  \label{eq:model.fourier}
    \begin{array}{ccc}
      \partial_z U &=& (j\omega L + R) I \\
      \partial_z I &=& (j\omega C  + G) U \\  
    \end{array}
\end{equation}
for the latter we denote by $Z$ the impedance matrix defined by $Z=(j\omega L + R)$ and $Y=(j\omega C  + G)$. The main concern of this paper is to compute $L$ with desired properties, independantly \todo{ ? } from the number of conductors, $C$ is straightforwardly deduced.

In the following chapters, we give the steps to compute the matrices at the continuous scale. We then suggest a way to enforce the symmetry and the definite positive conditions.
\subsection*{The single armour case}
\todo[inline]{rewrite from helluy's doc all the blabla}

\subsection*{The multiple armour/cable case}
\todo[inline]{rewrite from helluy's doc all the blabla}

\subsection*{Symmetry enforcement by weak formulation}


%%% Local Variables:
%%% mode: latex
%%% TeX-master: "../report"
%%% End:
