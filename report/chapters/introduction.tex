In this paper we are interested in  $N+1$ circular conductors $\{w_i\}_{i=0,..,N}$, considered in the $x,y$-plane, with respective radii $\{r_i\}_{i=0,..,N}$ and centers $\{X_i\}_{i=0,..,N}$. We mean by conductor \todo{figure?} , either a cable or a set of cables traveled by an electric \todo{a ameliorer ?} current in the $z$-axis, perpendicular to the plane. The top conductor $w_0$ is to be thought as the largest armour containing all the other conductors $w_i \, (i \ne 0)$. When the current travels over the cables, the transmission model allow us to determine the coupling between the cables. The model is a set of differential equations where the unknowns are the intensity $U(t,z)$ and the electric current $I(t,z)$, functions of time and space. The quantities are linked with coupling matrices $C$ and $L$ who has to respect properties such as positiveness and symmetry according to the physics. The main point of this paper is to analyse such behaviour at the numerical scale and propose a way to compute those matrices. 


%%% Local Variables:
%%% mode: latex
%%% TeX-master: "../report"
%%% End:
